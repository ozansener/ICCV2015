\section{Introduction}

Human communication is highly structured. For instance, in the description an event or a procedure, a clear beginning, end, and certain steps between these two can be usually identified. Parsing such communication into steps in a human-like manner is the key to understanding human activities. In this paper, we propose a method for decomposing a video into semantically meaningful steps in a fully unsupervised manner.
 
Language and vision are the two main modalities employed for this structured communication by humans. This is reflected in the fact that verbal (e.g., Do-It-Yourself books) and visual (e.g., instructional YouTube videos\footnote{YouTube has 281.000 videos for \emph{"How to tie a bow tie"}}) are the two main ways of explaining ``how-to'' perform a certain task or describing an event.  These two modalities often provide different, but correlating and complementary information. We utilize these two modalities (video frames and imperfect automatically generated subtitles) in a unified manner in our framework; we qualitatively and quantitatively argue that a joint inference is crucial for a successful semantic parsing, particularly when no supervision is employed. Note that subtitles are available for all YouTube videos generated either via Automatic Speech Recognition (ASR) ($95\%$ of the videos) or by the user ($5\%$).

We evaluate the proposed approach on instructional videos from YouTube (e.g., ``Making panckage'', ``How to tie a bow tie'') as they typically have objective steps and provide concrete grounds for demonstrating a semantically meaningful parsing. These videos are often long and manifest a great deal of intra-class variability on the surface yet the underlying objective structure, similar to almost all human communications, remains in place. In principle, the proposed parsing method is applicable to any type of videos as long as they are composed of a set of steps between their beginning and end.

The output of our method can be seen as the semantic ``storyline'' of a rather long and complex video (see Fig. \ref{fig:teaser}). This storyline provides what particular steps are taking place in the video (\emph{what}), what their semantic meaning is (\emph{how}), and when they are occurring (\emph{when}). This method is also capable of putting multiple videos performing the same overall task in common ground (i.e., the semantic steps’ space) and capture their high-level similarity, and therefore, provide a \emph{categorical} storyline as well. (see Fig. \ref{fig:teaser}). 

In a nutshell, given a video, we capture the visual properties of object proposals from each frame as well as a histogram of keywords from the subtitle. We then employ a generative beta process mixture model, which identifies the semantic steps shared among the videos pertained to the same category based on both text and visual cues. The model also identifies the text keywords which were deemed highly related to the visuals information of each steps. We later learn a language model to provide a textual description of the semantic steps based on the identified keywords. 

This work is the first to provide a semantic storyline for a complex video and video category. We are also the first to approach this problem in a multimodal (joint language and vision) manner. In addition, our method is capable of providing a caption describing the steps. Our approach to captioning is fundamentally different from the majority of existing video/image-to-text work in two aspects: 1) the captions are generated in an unsupervised manner, 2) our captions are \emph{descriptions} of the semantic steps, yet they are inferred from \emph{narration} text. This is different from the existing captioning work as their reference data is also descriptive of the visual information, while narration text often provides complementary information to the visuals and is not necessarily descriptive of them.  

%Leaning the instructions of a novel non-trivial task is both a challenge and a necessity for both humans and autonomous systems. This necessity resulted in many community generated instruction collections \cite{wikiHow,eHow} and expert curated recipe books\cite{recipeBook1,recipeBook2}. However, this instructions are generally based on a language modality and explains a single way of performing the task although there are variety of ways. On the other hand, online video storage services are full of unstructured instructional videos\footnote{YouTube has 281.000 videos for \emph{"How to tie a bow tie"}} covering variety of ways, environment conditions and view angles. Although there have been many successful attempts in detecting activities from videos \cite{act1,act2}, structural representation of such a large and useful video collection is not possible. In this paper, we focus on joint semantic representation of YouTube videos as a response to a single query. We specifically study the unsupervised joint-detection of the activities from a collection of YouTube videos.

%Understanding of the instructional videos, requires the careful processing of two complementary modalities namely language and the vision.  Luckily the target domain, YouTube videos, has unstructured subtitles as well. They are either generated by the content developer (5\% of the time) or automatically generated by using the Automatic Speech Recognition (ASR) software. The main limitations of the existing activity detection literature for this problem is scalability and representation level. Existing approaches are mainly supervised and requires extensive training set which is not tractable in the scale of YouTube videos. Moreover, current activity detection research focuses on the low-level visual features. However, such videos in the wild have objects with completely different texture and shape characteristics from wide range of views. Instead, we focus on extracting high-level visual semantic representations and using salient words occurring among the videos.

%We rely on the assumption that the videos collected as the response of a same instructional query, share similar activities performed by the similar objects. We start with the independent processing of the videos in order to create a large collection of visual object proposals and words. After the proposal generation, we jointly process the proposal collections and words to detect the visual objects and words which can be used to represent the unstructured information. Since we rely on high-level information instead of the low-level features, the resulting objects represent the semantic information instead of visual characteristic. By using the extracted objects, we compute the holistic representation of the multi-modal information in each frame.

%Moving from frame-wise visual understanding to activity understanding, requires the joint processing of all the videos with the temporal information. In order to exploit the temporal information, we model each video as a Hidden Markov Model using state space of activities. Since we assume that the videos share some of the activities and we have no supervision, we use a model based on \emph{beta process mixture model}. Our model jointly learn the activities and detect them in the videos. Moreover, it does not require prior knowledge over the number of activities.
